\documentclass[12 pt]{article}        	%sets the font to 12 pt and says this is an article (as opposed to book or other documents)
\usepackage{amsfonts, amssymb}					% packages to get the fonts, symbols used in most math
\usepackage{graphicx}
  
%\usepackage{setspace}               		% Together with \doublespacing below allows for doublespacing of the document

\oddsidemargin=-0.5cm                 	% These three commands create the margins required for class
\setlength{\textwidth}{6.5in}         	%
\addtolength{\voffset}{-20pt}        		%
\addtolength{\headsep}{25pt}           	%



\pagestyle{myheadings}                           	% tells LaTeX to allow you to enter information in the heading
\markright{Andrew Mayo\hfill \today \hfill}  
																									% and put the proposition number from the book
                                                	% LaTeX will put your name on the left, the date the paper 
                                                	% is generated in the middle 
                                                 	% and a page number on the right



\newcommand{\eqn}[0]{\begin{array}{rcl}}%begin an aligned equation - allows for aligning = or inequalities.  Always use with $$ $$
\newcommand{\eqnend}[0]{\end{array} }  	%end the aligned equation

%\doublespacing                         	% Together with the package setspace above allows for doublespacing of the document

\newcommand{\qed}[0]{$\square$}        	% make an unfilled square the default for ending a proof

\begin{document}												% end of preamble and beginning of text that will be printed 
\textbf{1 (a)} Note that $\frac{1}{1 + e^z} = 1 - \sigma(z) $ and $ \frac{\partial \sigma(z)}{\partial z} = \sigma(z) (1 - \sigma(z)$. 
\[
  \frac{\partial{E(w)}}{\partial w_j} = 
  \sum_{i=1}^{N} (x_j)^{(i)} y^{(i)} (1 - \sigma(w^T x^{(i)}))
  - (x_j)^{(i)} \sigma(w^T x^{(i)}) 
  + (x_j)^{(i)} y^{(i)} \sigma(w^T x^{(i)})
\]
\[
  = \sum_{i=1}^{N} (x_j)^{(i)} y^{(i)} 
  - (x_j)^{(i)} \sigma(w^T x^{(i)}) 
\]
\[
  \frac{\partial^2 E(w)}{\partial (w_j)^2}
  = - \sum_{i=1}^{N} (x_j)^{(i)} \sigma(w^T x^{(i)}) (1 - \sigma(w^T x^{(i)})) (x_j)^{(i)}
\]
\[
  \frac{\partial^2 E(w)}{\partial w_j w_k}
  = - \sum_{i=1}^{N} (x_j)^{(i)} \sigma(w^T x^{(i)}) (1 - \sigma(w^T x^{(i)})) (x_k)^{(i)}
\]
Let $ X \in \mathbb{R}^{n \times m} $ be the design matrix 
and $ w \in \mathbb{R}^m $ be the weight vector, 
where $ n $ is the number of observations
and $ m $ is the number of features. 
We can express the second-order partial derivatives in matrix form as 
\[
  X^T diag[ \sigma(Xw) (1 - \sigma(Xw)) ] X
\]
which gives us our Hessian matrix.
\end{document}
